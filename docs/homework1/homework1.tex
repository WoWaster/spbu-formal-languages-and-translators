\documentclass[12pt, oneside]{memoir}
\input{../preamble.tex}
\title{ТФЯиТ. Домашняя работа №1}
\author{Николай Пономарев, группа 21.Б10-мм}

\begin{document}
\maketitle

\section*{Упражнение I-1.1}

\begin{solution}
    {Функция \[K(i, j) = \frac{(i + j - 1) (i + j - 2)}{2} + j.\]}
    {Найти обратные к ней функции $I$ и $J$, а так же предъявить код для найденных функций.}
    {Найдем формулы для $I$ и $J$, а затем напишем код.}
    Для начала рассмотрим случай $i = 1$, тогда функция $K$ примет вид:
    \[K(j) = \frac{j (j - 1)}{2} + j,\]
    равносильными преобразованиями приведем её к следующему виду:
    \[j^2 + j - 2K = 0.\]
    Корнями этого уравнения относительно $j$ будут:
    \[j_{1, 2} = \frac{-1 \pm \sqrt{1 + 8 K}}{2}\]
    Знаем, что минимальное значение $K = 1$, поэтому можем вычислять только корень
    \[j_1 = \frac{-1 + \sqrt{1 + 8 K}}{2}\]
    В таком случае, если $j_1 \in \N$, то мы нашли решение задачи вида
    \begin{align*}
        i & = 1 & j & =\frac{-1 + \sqrt{1 + 8 K}}{2}
    \end{align*}

    Теперь рассмотрим ситуацию, когда $j_1 \not\in \N$.
    Знаем, что искомую функцию можно переписать как $K(i,j) = N + j$, где
    \[N = \frac{n(n-1)}{2}\]
    и $N$~--- номер ближайшей пары с целым $j_1$, а $n$~--- число диагоналей.
    Тогда с помощью равносильных преобразований получаем, что
    \begin{gather*}
        i = n - j + 2 \\
        j = K - N
    \end{gather*}
    Кроме того, справедливы соотношения
    \[n < N \le K\]

    Код на языке Haskell:
\end{solution}
\begin{minted}{haskell}
helper :: Int -> Int -> (Int, Int)
helper nBig k =
    if equation == 0
        then
            if nBig /= k
                then (integerRoot + 2 - k + nBig, k - nBig)
                else (1, integerRoot)
        else helper (nBig - 1) k
    where
    root = (-1 + sqrt (1.0 + 8.0 * fromIntegral k)) / 2.0
    integerRoot = truncate root
    equation = integerRoot ^ 2 + integerRoot - 2 * nBig

ij :: Int -> (Int, Int)
ij k = helper k k
\end{minted}

\section*{Упражнение I-1.2}
\begin{solution}
    {Функция $\hat{J} (w, x, y) = J(w, J(x, y))$.}
    {Значения $w, x, y$, если $\hat{J} = 1000$.}
    {Вычислим с помощью функций, полученных выше.}
    Применив функцию выше, получим, что $w = 36$, а $J(x, y) = 10$.
    Применив функцию повторно, получим $x = 1$, а $y = 4$.
\end{solution}

\section*{Упражнение I-1.3}
\begin{solution}
    {Рекурсивный язык $L$.}
    {Перенумеровать предложения языка $L$.}
    {Предъявим описание процедуры.}
    Поскольку язык $L$ является рекурсивным, то существует алгоритм его распознавания, кроме того существует процедура порождения данного языка.

    Заведем счётчик $i = 0$.
    И процедура будет выглядеть следующим образом:
    \begin{enumerate}
        \item Порождаем предложение $l$ языка $L$;
        \item Проверяем, что порожденное предложение принадлежит языку $L$.
              В силу того, что мы имеем алгоритм распознавания, этот шаг всегда завершится;
        \item Если предложение $l$ принадлежит языку $L$, то присваиваем ему номер $i$ и увеличиваем значение счётчика на единицу;
        \item Возвращаемся к шагу 1.
    \end{enumerate}
\end{solution}

\section*{Упражнение I-1.4}
\begin{solution}
    {Процедура перечисления множества целых в монотонном порядке.}
    {Доказать, что это множество рекурсивно.}
    {Предъявим описание процедуры.}
    Пусть $n$~--- число, принадлежность множеству которого мы хотим проверить.
    \begin{enumerate}
        \item Порождаем новый элемент $m$ из множества целых;
        \item Если $m = n$, то алгоритм выдает ответ \enquote{да};
        \item Если $m < n$, то то переходим к шагу 1.;
        \item Если $m > n$, то в силу монотонности процедуры перечисления, получить меньшее число $m$ не представляется возможным и алгоритм заканчивает работу с ответом \enquote{нет}.
    \end{enumerate}
\end{solution}

\section*{Упражнение I-1.5}
\begin{solution}
    {Конечное множество $M$.}
    {Показать, что множество $M$ рекурсивно.}
    {Предъявим описание процедуры.}
    Хотим проверить, что элемент $x$ принадлежит множеству $M$.

    Для поверки достаточно проитерироваться по множеству, проверяя не совпал на элемент $x$ с текущим просматриваемым элементом из $M$.
    Если случилось совпадение, то алгоритм выдаст ответ \enquote{да}, в ином случае \enquote{нет}.

    Поскольку множество конечно, любой просмотр множества выполняется за конечное время и всегда завершается.
\end{solution}

\end{document}
