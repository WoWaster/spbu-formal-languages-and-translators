\documentclass[12pt, oneside]{memoir}
\input{../preamble.tex}
\title{ТФЯиТ. Домашняя работа №3}
\author{Николай Пономарев, группа 21.Б10-мм}

% \tikzexternalize
\semiisopage
\begin{document}
\maketitle

\section*{Упражнение I-3.2}

\begin{solution}
    {Грамматика, порождающая числа без знака языка Паскаль:
        \begin{bnf*}
            \bnfprod{NUMBER}{\bnfts{d} \bnfor \bnfts{d} \bnfpn{FRACTION} \bnfor \bnfts{d} \bnfpn{EXPONENT}} \\
            \bnfmore{\bnfor \bnfts{d} \bnfpn{FLOAT} \bnfor \bnfts{d} \bnfpn{NUMBER}} \\
            \bnfprod{FRACTION}{\bnfts{.} \bnfpn{INT}} \\
            \bnfprod{INT}{\bnfts{d} \bnfor \bnfts{d} \bnfpn{INT}} \\
            \bnfprod{EXPONENT}{\bnfts{e} \bnfpn{INTEGER}} \\
            \bnfprod{INTEGER}{\bnfts{+} \bnfpn{INT} \bnfor \bnfts{-} \bnfpn{INT} \bnfor \bnfts{d} \bnfor \bnfts{d} \bnfpn{INT}} \\
            \bnfprod{FLOAT}{\bnfts{.} \bnfpn{FREXP}} \\
            \bnfprod{INT}{\bnfts{d} \bnfpn{EXPONENT} \bnfor \bnfts{d} \bnfpn{FREXP}}
        \end{bnf*}
    }
    {Конечный автомат по данной грамматике.}
    {Предъявим NFA.}
    \begin{center}
        \begin{tikzpicture}[on grid,auto,node distance=4cm,every state/.style={font=\tiny}]
            \node[state, initial above] (number) {NUMBER};
            \node[state] (fraction) [below left=of number] {FRACTION};
            \node[state] (int) [below=of fraction] {INT};
            \node[state] (exponent) [below=of number] {EXPONENT};
            \node[state] (integer) [below=of exponent] {INTEGER};
            \node[state] (float) [below right=of number] {FLOAT};
            \node[state] (frexp) [below=of float] {FREXP};
            \node[state, accepting] (accepting) [below right=of frexp] {ACC};

            \path[->] (number) edge [loop left] node {$d$} ();
            \path[->] (number) edge node {$d$} (fraction);
            \path[->] (fraction) edge node {$\cdot$} (int);
            \path[->] (int) edge [loop left] node {$d$} ();
            \path[->] (number) edge node {$d$} (exponent);
            \path[->] (exponent) edge node {$e$} (integer);
            \path[->] (integer) edge [bend left=30] node [above]{$+$} (int);
            \path[->] (integer) edge [bend right=30] node [above]{$-$} (int);
            \path[->] (integer) edge node [above]{$d$} (int);
            \path[->] (number) edge node {$d$} (float);
            \path[->] (float) edge node {$\cdot$} (frexp);
            \path[->] (frexp) edge [loop right] node {$d$} ();
            \path[->] (frexp) edge node {$d$} (exponent);
            \path[->] (number) edge [bend left = 60] node {$d$} (accepting);
            \path[->] (integer) edge node {$d$} (accepting);
            \path[->] (int) edge [bend right = 40] node {$d$} (accepting);
        \end{tikzpicture}
    \end{center}
\end{solution}


\section*{Упражнение I-3.3}
\begin{solution}
    {Конечный автомат из упражнения I-3.2.}
    {Построить детерминированный автомат на основе NFA из упражнения I-3.2.}
    {Воспользуемся алгоритмом из теоремы 3.4.}
    Для начала формально опишем NFA из упражнения I-3.2.
    $M = (Q, \Sigma, \delta, q_0, F)$, где
    \begin{align*}
        Q = \{          & \mathrm{NUMBER}, \mathrm{FRACTION}, \mathrm{EXPONENT}, \mathrm{FLOAT}, \\
                        & \mathrm{INT}, \mathrm{INTEGER}, \mathrm{FREXP}, \mathrm{ACC}\}         \\
        \Sigma = \{     & d, e, ., +, -\}                                                        \\
        q_0         ={} & \mathrm{NUMBER}                                                        \\
        F = \{          & \mathrm{ACC}\}
    \end{align*}
    и набор отображений $\delta$:
    \begin{align*}
        \delta(\mathrm{NUMBER}, d)         & = \{\mathrm{NUMBER}, \mathrm{FRACTION}, \mathrm{EXPONENT}, \\
                                           & \mathrm{FLOAT},  \mathrm{ACC}\}                            \\
        \delta(\mathrm{FRACTION}, .)       & = \{\mathrm{INT}\}                                         \\
        \delta(\mathrm{INT}, d)            & = \{\mathrm{INT}, \mathrm{ACC}\}                           \\
        \delta(\mathrm{EXPONENT}, e)       & = \{\mathrm{INTEGER} \}                                    \\
        \delta(\mathrm{INTEGER}, \{+, -\}) & = \{\mathrm{INT}\}                                         \\
        \delta(\mathrm{INTEGER}, d)        & = \{\mathrm{INT}, \mathrm{ACC}\}                           \\
        \delta(\mathrm{FLOAT}, .)          & = \{\mathrm{FREXP}\}                                       \\
        \delta(\mathrm{FREXP}, d)          & = \{\mathrm{EXPONENT}, \mathrm{FREXP}\}
    \end{align*}
    Будем использовать алгоритм теоремы 3.4 для получения DFA $M' = (Q', \Sigma, \delta', q_0', F')$, где $\Sigma$ как выше.
    Для начала возьмём $Q' = 2^Q$, и положим $q_0' = [q_0]$, а так же $F' = \{E \in Q' : F \subset E\}$, далее уменьшим множества $Q'$ и $F'$.

    Переобозначим для краткости необходимые нам множества:
    \begin{align*}
        q_0 & = [\mathrm{NUMBER}]                                                                                 & q_4 & = [\mathrm{INT}, \mathrm{ACC}, \mathrm{EXPONENT}, \mathrm{FREXP}] \\
        q_1 & = \mathrlap{[\mathrm{NUMBER}, \mathrm{FRACTION}, \mathrm{EXPONENT}, \mathrm{FLOAT},  \mathrm{ACC}]}                                                                           \\
        q_2 & = [\mathrm{INT, FREXP}]                                                                             & q_5 & = [\mathrm{INT}, \mathrm{ACC}]                                    \\
        q_3 & = [\mathrm{INTEGER}]                                                                                & q_6 & = [\mathrm{INT}]
    \end{align*}
    Теперь построим набор отображений $\delta'$:
    \begin{align*}
        \delta'(q_0, d) & = q_1 & \delta'(q_3, d)        & = q_5 \\
        \delta'(q_1, d) & = q_1 & \delta'(q_3, \{+, -\}) & = q_6 \\
        \delta'(q_1, .) & = q_2 & \delta'(q_4, d)        & = q_4 \\
        \delta'(q_1, e) & = q_3 & \delta'(q_4, e)        & = q_3 \\
        \delta'(q_2, d) & = q_4 & \delta'(q_5, d)        & = q_5 \\
                        &       & \delta'(q_6, d)        & = q_5
    \end{align*}
    Получили DFA $M' = (Q', \Sigma, \delta', q_0', F')$, где $Q' = \{q_0, q_1, q_2, q_3, q_4, q_5, q_6\}$, $q_0' = q_0$ и $F' = \{q_1, q_4, q_5\}$.
\end{solution}

\section*{Упражнение I-3.4}
\begin{solution}
    {DFA из упражнения I-3.3.}
    {Построить эквивалентную автоматную грамматику.}
    {Воспользуемся алгоритмом из теоремы 3.6.}
    Будем использовать DFA $M' = (Q', \Sigma, \delta', q_0', F')$, построенный выше.
    Пусть $G = (V_N, V_T, P, S)$, где $V_N = Q'$, $V_T = \Sigma$, $S = q_0$ и набор правил:
    \begin{align*}
        P = \{ & q_0 \to d q_1, q_0 \to d,                               \\
               & q_1 \to d q_1, q_1 \to d, q_1 \to . q_2, q_1 \to e q_3, \\
               & q_2 \to d q_4, q_2 \to d,                               \\
               & q_3 \to d q_5, q_3 \to d, q_3 \to + q_6, q_3 \to - q_6, \\
               & q_4 \to d q_4, q_4 \to d, q_4 \to e q_3,                \\
               & q_5 \to d q_5, q_5 \to d,                               \\
               & q_6 \to d q_5, q_6 \to d\}.
    \end{align*}
\end{solution}

\section*{Упражнение I-3.5}
\begin{solution}
    {Автоматная грамматика из упражнения I-3.4.}
    {Построить по ней эквивалентное регулярное выражение.}
    {Воспользоваться грамматикой и нотацией из примера 3.5.}
    Получается регулярное выражение: $d^+[.d^+][e[+,-]d^+]$.
\end{solution}

% \section*{Упражнение I-3.6}
% \begin{solution}
%     {DFA $M$, полученный в упражнении I-3.3.}
%     {Определить является ли он минимальным.}
%     {Воспользуемся алгоритмом теоремы 3.3.}

% \end{solution}
\newpage
Грамматика $\lambda$-исчисления:
\begin{align*}
    x \in V                & \implies x \in \Lambda             \\
    M, N \in \Lambda       & \implies (M\ N) \in \Lambda        \\
    M \in \Lambda, x \in V & \implies (\lambda x.M) \in \Lambda
\end{align*}

Или
\[\Lambda ::= V \mid (\Lambda\ \Lambda) \mid (\lambda V.\Lambda)\]

\[\forall \ \exists\]
\end{document}
