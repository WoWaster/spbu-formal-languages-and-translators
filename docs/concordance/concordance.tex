\documentclass[12pt, oneside]{memoir}
\input{../preamble.tex}
\title{ТФЯиТ. КР №2}
\author{Николай Пономарев, группа 21.Б10-мм}

% \tikzexternalize

\begin{document}
\maketitle

\chapter{Предметный указатель}
Код основной функции программы на языке Haskell:
\inputminted[breaklines, firstline=13, lastline=33]{haskell}{../../app/concordance/Main.hs}

Примеры работы программы:
\begin{minted}[breaklines]{bash}
> cabal run concordance -- examples/concordance/ae_fond_kiss.txt
("Ae",[1,2,23,24])
("But",[12])
("Dark",[8])
("Deep",[3,25])
("Fare",[19,20])
("Fortune",[5])
("Had",[14,15])
("I",[3,4,10,25,26])
("Love",[13])
("Me",[7])
("Naething",[11])
("Nancy",[11])
("Never",[16])
("Peace",[22])
("Thine",[21])
("Warring",[4,26])
("We",[17])
("While",[6])
("Who",[5])
("alas",[24])
("and",[1,2,4,13,19,20,21,22,23,26])
("around",[8])
("be",[21])
("been",[17])
("benights",[8])
("best",[20])
("blame",[10])
("blindly",[15])
("broken",[17])
("but",[13])
("cheerfu",[7])
("could",[11])
("dearest",[20])
("despair",[8])
("enjoyment",[22])
("fairest",[19])
("fancy",[10])
("fareweel",[2,24])
("first",[19])
("fond",[1,23])
("forever",[2,13,24])
("grieves",[5])
("groans",[4,26])
("had",[17])
("heart",[3,25])
("her",[12,12,13])
("him",[5,6])
("hope",[6])
("ilka",[21])
("in",[3,25])
("joy",[21])
("kindly",[14])
("kiss",[1,23])
("leaves",[6])
("lights",[7])
("lov",[14,15])
("love",[12,13,22])
("me",[7,8])
("met",[16])
("my",[10,11])
("nae",[7])
("ne",[10,17])
("never",[14,15,16])
("of",[6])
("parted",[16])
("partial",[10])
("pleasure",[22])
("pledge",[3,25])
("resist",[11])
("sae",[14,15])
("say",[5])
("see",[12])
("sever",[1,23])
("shall",[5])
("she",[6])
("sighs",[4,26])
("star",[6])
("tears",[3,25])
("that",[5])
("the",[6])
("thee",[3,4,19,20,25,26])
("then",[1,2,23])
("thou",[19,20])
("to",[12,12])
("treasure",[21])
("twinkle",[7])
("wage",[4,26])
("was",[12])
("we",[1,14,15,23])
("weel",[19,20])
\end{minted}

\end{document}
