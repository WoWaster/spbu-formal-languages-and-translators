\documentclass[12pt, oneside]{memoir}
\input{../preamble.tex}
\title{ТФЯиТ. Домашняя работа №4}
\author{Николай Пономарев, группа 21.Б10-мм}

% \tikzexternalize
\begin{document}
\maketitle

Упражнения I-2.5, I-2.16 были включены в список упражнений после второй лекции, а упражнения I-3.2--I-3.6 были включены в список упражнений после третьей лекции, здесь они не приводятся.

\section*{Упражнение I-2.7}

\begin{solution}
    {Грамматика $G = (V_N, V_T, P, S)$, где $V_N = \{A, B, S\}$, $V_T = \{0, 1\}$ и
        \[P= \{S \to 0AB,\ B \to 01,\ 1B \to 0,\ A1 \to SB1,\ B \to SA,\ A0 \to S0B\}.\]}
    {Доказать, что $L(G)$ пуст.}
    {Рассуждения.}
    Новые нетерминалы не порождаются только двумя правилами: $B \to 01$ и $1B \to 0$.
    В таком случае, чтобы получить слово из языка можно только если сентенциальная форма не содержит нетерминала $A$.
    \enquote{Избавиться} от нетерминала $A$ можно используя два правила: $A1 \to SB1$ и $A0 \to S0B$.
    Однако данные правила порождают нетерминал $S$, из которого по единственному правилу $S \to 0AB$ всегда будет появляться нетерминал $A$.
    Поэтому мы будем получать сентенциальные формы с нетерминалами, но не слова из языка, поэтому язык грамматики $G$ пуст.
\end{solution}

\section*{Упражнение I-2.8}
\begin{solution}
    {$G$~--- грамматика, все продукции которой имеют форму $A \to xB$ и $A \to x$, где $A$ и $B$ ~--- нетерминалы, а $x$~--- строка терминалов.}
    {Доказать, что $L(G)$ может быть порожден регулярной грамматикой.}
    {Рассуждения.}
    Данные продукции практически удовлетворяют условию регулярной грамматики, только $x$ не один терминал, а строка терминалов.

    Пусть $n$~--- длина строки $x$ и $x$ представима в виде $x = x_1 x_2 \dots x_n$, где $x_i$ есть терминальный символ при $i = 1, \dots, n$.
    Построим новые правила вывода следующим образом:
    \begin{align*}
        A   & \to x_1 X_2     & A     & \to x_1 Y_2     \\
        X_2 & \to x_2 X_3     & Y_2   & \to x_2 Y_3     \\
            & \dots           & \dots                   \\
        X_i & \to x_i X_{i+1} & Y_i   & \to x_i Y_{i+1} \\
            & \dots           & \dots                   \\
        X_n & \to x_n         & Y_n   & \to x_n B
    \end{align*}
    Теперь получили правила вывода, соответствующие условиям регулярной грамматики, а значит доказали, что $L(G)$ может быть порожден регулярной грамматикой.
\end{solution}

\section*{Упражнение I-2.12}
\begin{solution}
    {КС-грамматика $G = (V_N, V_T, P, S)$, где $V_N = \{S, A, B\}$, $V_T = \{a, b\}$ и
        \begin{align*}
            P =\{ & S \to aB,\ S \to bA,\ A \to a,\ A \to aS,     \\
                  & A \to bAA,\ B \to b,\ B \to bS,\ B \to aBB\}.
        \end{align*}}
    {Порождаемый язык.}
    {Рассуждения.}

\end{solution}

% \section*{Упражнение I-3.6}
% \begin{solution}
%     {Автоматная грамматика из упражнения I-3.4.}
%     {Построить по ней эквивалентное регулярное выражение.}
%     {Воспользоваться грамматикой и нотацией из примера 3.5.}
% \end{solution}

\end{document}
