\documentclass[12pt, oneside]{memoir}
%%% PDF settings
\pdfvariable minorversion 7 % Set PDF version to 1.7.

%%% Fonts and language setup.
\usepackage{polyglossia}
% Setup fonts.
\usepackage{fontspec}
\setmainfont{CMU Serif}
\setsansfont{CMU Sans Serif}
\setmonofont{Fira Code}

\usepackage{microtype} % Add fancy-schmancy font tricks.
\usepackage[autostyle]{csquotes}

\usepackage{xcolor} % Add colors support.

%% Math
\usepackage{amsmath, amsfonts, amssymb, amsthm, mathtools} % Advanced math tools.
\usepackage{thmtools}
% \usepackage{unicode-math} % Allow TTF and OTF fonts in math and allow direct typing unicode math characters.
\usepackage{euler-math}
\unimathsetup{
    warnings-off={
            mathtools-colon,
            mathtools-overbracket
        }
}
% \setmathfont{Latin Modern Math} % default
% \setmathfont[range={\setminus,\varnothing,\smashtimes}]{Asana Math}
% \setmainfont{CMU Concrete}
\setmainfont{DejaVu Serif}

%%% Images
\usepackage{graphicx}
\graphicspath{{figures/}}
\usepackage{import}

%%% Polyglossia setup after (nearly) everything as described in documentation.
\setdefaultlanguage{russian}
\setotherlanguage{english}


%%% Custom commands
\newcommand{\R}{\mathbb{R}}
\newcommand{\N}{\mathbb{N}}
\newcommand{\Z}{\mathbb{Z}}
\newcommand{\Q}{\mathbb{Q}}
\newcommand{\C}{\mathbb{C}}
\newcommand{\id}{\mathrm{id}}
\AtBeginDocument{\renewcommand{\leq}{\leqslant}}
\AtBeginDocument{\renewcommand{\geq}{\geqslant}}
\AtBeginDocument{\renewcommand{\Re}{\operatorname{Re}}}
\AtBeginDocument{\renewcommand{\Im}{\operatorname{Im}}}
\AtBeginDocument{\renewcommand{\phi}{\varphi}}
\AtBeginDocument{\renewcommand{\epsilon}{\varepsilon}}

%%% theorem-like envs
\theoremstyle{definition}

\declaretheoremstyle[spaceabove=0.5\topsep,
    spacebelow=0.5\topsep,
    headfont=\bfseries\sffamily,
    bodyfont=\normalfont,
    headpunct=.,
    postheadspace=5pt plus 1pt minus 1pt]{myStyle}
\declaretheoremstyle[spacebelow=\topsep,
    headfont=\bfseries\sffamily,
    bodyfont=\normalfont,
    headpunct=.,
    postheadspace=5pt plus 1pt minus 1pt,]{myStyleWithFrame}
\declaretheoremstyle[spacebelow=\topsep,
    headfont=\bfseries\sffamily,
    bodyfont=\normalfont,
    headpunct=.,
    postheadspace=5pt plus 1pt minus 1pt,
    qed=\blacksquare]{myProofStyleWithFrame}

\usepackage[breakable]{tcolorbox}
\tcbset{sharp corners=all, colback=white}
\tcolorboxenvironment{theorem}{}
\tcolorboxenvironment{theorem*}{}
\tcolorboxenvironment{axiom}{}
\tcolorboxenvironment{assertion}{}
\tcolorboxenvironment{lemma}{}
\tcolorboxenvironment{proposition}{}
\tcolorboxenvironment{corollary}{}
\tcolorboxenvironment{definition}{}
\tcolorboxenvironment{proofReplace}{toprule=0mm,bottomrule=0mm,rightrule=0mm, colback=white, breakable }

\declaretheorem[name=Теорема, numberwithin=chapter, style=myStyleWithFrame]{theorem}
\declaretheorem[name=Теорема, numbered=no, style=myStyleWithFrame]{theorem*}
\declaretheorem[name=Аксиома, sibling=theorem, style=myStyleWithFrame]{axiom}
\declaretheorem[name=Преположение, sibling=theorem, style=myStyleWithFrame]{assertion}
\declaretheorem[name=Лемма, sibling=theorem, style=myStyleWithFrame]{lemma}
\declaretheorem[name=Предложение, sibling=theorem, style=myStyleWithFrame]{proposition}
\declaretheorem[name=Следствие, numberwithin=theorem, style=myStyleWithFrame]{corollary}

\declaretheorem[name=Определение, numberwithin=chapter, style=myStyleWithFrame]{definition}
\declaretheorem[name=Свойство, numberwithin=chapter, style=myStyle]{property}
\declaretheorem[name=Свойства, numbered=no, style=myStyle]{propertylist}

\declaretheorem[name=Пример, numberwithin=chapter, style=myStyle]{example}
\declaretheorem[name=Замечание, numbered=no, style=myStyle]{remark}

\declaretheorem[name=Доказательство, numbered=no, style=myProofStyleWithFrame]{proofReplace}
\renewenvironment{proof}[1][\proofname]{\begin{proofReplace}}{\end{proofReplace}}
\declaretheorem[name=Доказательство, numbered=no, style=myProofStyleWithFrame]{longProof}

\NewDocumentEnvironment{solution}{mmm +b}{
    \textbf{ДАНО:} #1

    \textbf{НАЙТИ:} #2

    \textbf{МЕТОД РЕШЕНИЯ:} #3

    \textbf{РЕШЕНИЕ:} #4
}{}

\usepackage{minted}
\usepackage[altpo, epsilon]{backnaur}
\usepackage{tikz}
\usetikzlibrary{external, automata, positioning}

%%% Memoir settings
\chapterstyle{section}
\gdef\clearforchapter{}
\semiisopage
% \setlength{\headheight}{2\baselineskip}

%%% HyperRef
\usepackage{hyperref}

\title{ТФЯиТ. Домашняя работа №2}
\author{Николай Пономарев, группа 21.Б10-мм}

\begin{document}
\maketitle

\section*{Упражнение I-2.1}

\begin{solution}
    {Грамматика $G = (V_N, V_T, P, S)$, где $V_N = \{S, A, B\}, \\ V_T = \{0, 1\}$,
        \begin{align*}
            P = \{
             & S \to 0A, \   S \to 1B, \   S \to 0,    \\
             & A \to 0A,\    A \to 0S,\    A \to 1B,   \\
             & B\to 1B, \    B\to 1,   \    B\to 0 \}.
        \end{align*}
    }
    {Язык, порождаемый данной грамматикой.}
    {Рассуждения.}
    Для начала рассмотрим случи, когда нетерминал $B$ никогда не появляется в выводе.
    Тогда единственное конечное правило $S \to 0$.
    И мы можем получать цепочки из $0$ любой длины, кроме 2.

    Если нетерминал $B$ хотя бы раз появился в выводе, то последним символом может быть $0$ или $1$.
    Кроме того, из рассуждений выше мы можем сгенерировать ноль или больше символов $0$, после чего если хоть раз выбирается нетерминал $B$, то появляется хотя бы одна $1$, а затем мы можем генерировать только $1$, либо закончить вывод.
    В таком случае получаются цепочки вида $0^* 1^+ (0,1)$.

    Итого, $L(G) = \{0^n : n \in \N \setminus \{2\}\} \cup 0^* 1^+ (0,1)$.
\end{solution}


\section*{Упражнение I-2.2}
\begin{solution}
    {Язык $L = \{w \mid w \in \{0,1\}^* \}$, где $w$ не содержит двух последовательных единиц.}
    {Регулярную грамматику, порождающую этот язык.}
    {Предъявим грамматику.}
    $G = (V_N, V_T, P, S)$, где $V_N = \{S, A\}$, $V_T =\{0, 1\}$ и
    \begin{align*}
        P \{ & S \to 0, S \to 1, S \to 0S,   \\
             & A \to 0, S \to 1A, A \to 0S\}
    \end{align*}
\end{solution}

% \section*{Упражнение I-2.3}
% \begin{solution}
%     {КС грамматика $G = (\{S, A\}, \{a, b\}, P, S)$, где $P = \{S\to aAS, \ S \to a, \ A \to SbA, \ A \to ba, \ A \to SS\}$ из примера $2.7$.}
%     {Неформально описать слова, порождающиеся этой грамматикой.}
%     {Рассуждения.}

% \end{solution}

\section*{Упражнение I-2.4}
\begin{solution}
    {Алгоритм теоремы $2.2$.}
    {Проверить принадлежность слов "$abaa$", "$abbb$", "$baaba$" к грамматике из примера $2.7$.}
    {Воспользоваться алгоритмом теоремы $2.2$.}
    Построим множества $T_m$.
    Будем строить для $n = 5$, так как это длина самого длинного слова, которое нам необходимо проверить.
    Общее условие на $T_m$ в нашем случае:
    \[T_m = \left\{\alpha : S \xRightarrow[G]{i} \alpha, i \le m, \alpha \in V^*, |\alpha| \le 5\right\}\]
    Начнем построение:
    \begin{align*}
        T_0 & = \{S\}                                             \\
        T_1 & = T_0 \cup \{aAS, a\}                               \\
        T_2 & = T_1 \cup \{aSbAS, abaS, aSSS, aAaAS, aAa\}        \\
        T_3 & = T_2 \cup \{aabAS, aSbAa, abaa, aaSS, aSaS, aSSa\} \\
        T_4 & = T_3 \cup \{aabAa, aaaS, aaSa, aSaa\}              \\
        T_5 & = T_4 \cup \{aaaa\}
    \end{align*}
    Более построить множеств нельзя, иначе нарушается требования на длину.
    Выпишем множество $T_5$:
    \begin{align*}
        T_5 = \{ & S, aAS, a, aSbAS, abaS, aSSS, aAaAS, aAa, aabAS, aSbAa,  \\
                 & abaa, aaSS, aSaS, aSSa,aabAa, aaaS, aaSa, aSaa, aaaa  \}
    \end{align*}
    Таким образом, можем заключить, что только $abaa \in L(G)$.
\end{solution}

\section*{Упражнение I-2.5}
\begin{solution}
    {$G$~--- КС или регулярная грамматика.}
    {Выяснить, можно ли улучшить ограничение на $m$ в теореме $2.2$.}
    {Рассуждения.}
    В случае если $G$~--- КС-грамматика, улучшения не получить, т.к. ограничений на длину выводимой цепочки нет.

    В случае если $G$~--- регулярная грамматика, можно достичь улучшения.
    Если $G = (V_N, V_T, P, S)$ и она регулярная грамматика, то правила имеют вид $A \to aB$ или $A \to a$, где $a \in V_T$, $A,B \in V_N$.
    Это значит, что на каждом шаге мы либо не увеличиваем длину цепочки, либо увеличиваем на единицу.
    Следовательно, мы получаем ограничение вида $m \le n$.
\end{solution}

\section*{Упражнение I-2.16}
\begin{solution}
    {Неукорачивающая грамматика $G = (V_N, V_T, P, S)$, где $V_N = \{S, B, C\}$, $V_T = \{a,b,c\}$ и
        \begin{align*}
            P = \{ & S \to aSBC,\ S \to aBC,\ CB \to BC               \\
                   & aB \to ab,\ bB \to bb,\ bC \to bc,\ cC \to cc\}.
        \end{align*}}
    {Построить НС-грамматику, эквивалентную данной.}
    {Воспользуемся алгоритмом приведения неукорачивающей грамматики к НС-грамматике.}
    Все правила, кроме $CB \to BC$ уже соответствуют требованиям НС-грамматики.
    Пусть $X$ и $Y$~--- новые нетерминалы, тогда заменим правило $CB \to BC$ на следующие четыре правила:
    \begin{gather*}
        CB \to XB \qquad XB \to XY \qquad XY \to BY \qquad BY \to BC
    \end{gather*}
    Все они удовлетворяют требованиям НС-грамматик.
    Итого, получили грамматику $G' = (V_N, V_T, P, S)$, где $V_N = \{S, B, C, X, Y\}$, $V_T = \{a,b,c\}$ и
    \begin{align*}
        P = \{ & S \to aSBC,\ S \to aBC,\ aB \to ab,                \\
               & bB \to bb,\ bC \to bc,\ cC \to cc,                 \\
               & CB \to XB,\ XB \to XY, \ XY \to BY, \ BY \to BC\}.
    \end{align*}
\end{solution}

\end{document}
