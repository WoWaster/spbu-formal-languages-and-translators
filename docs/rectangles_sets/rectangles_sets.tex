\documentclass[12pt, oneside]{memoir}
\input{../preamble.tex}
\title{ТФЯиТ. КР №1}
\author{Николай Пономарев, группа 21.Б10-мм}

% \tikzexternalize

\begin{document}
\maketitle

\chapter{Прямоугольники}
Код основной функции программы на языке Haskell:
\inputminted[breaklines, firstline=12, lastline=19]{haskell}{../../app/rectangles/Main.hs}

Примеры работы программы:
\begin{minted}[]{bash}
> stack run rectangles -- 7 7
784
> stack run rectangles -- 3 1
6
> stack run rectangles -- 16 20
28560
\end{minted}

\chapter{Множества}
Код основной функции программы на языке Haskell:
\inputminted[breaklines, firstline=8, lastline=46]{haskell}{../../app/sets/Main.hs}

Примеры работы программы:
\begin{minted}[breaklines]{bash}
> stack run sets -- examples/sets/sets1.txt examples/sets/sets2.txt
Union of sets: "
    ',-.?ABCDEFGHILMOPQSTVWabcdefghiklmnopqrstuvwxyz’"
Intersection of sets: "
    ,-.ADGIOSTWabcdefghiklmnoprstuvwxy"
Difference of sets: "BCEFLMPQVq’"
Size of alphabet 1 47
Size of alphabet 2 41
Size of set1 \ set2 11
\end{minted}
\end{document}
